\documentclass[12pt, letterpaper]{article}
	\usepackage{amsmath,amssymb,amsthm,amsopn,amscd}
	\usepackage{mathtools}
	\usepackage{latexsym}
	\usepackage{graphicx,caption,subcaption}
	\usepackage{multirow}
	\usepackage[reftex]{theoremref}
	\usepackage{hyperref}
	\usepackage{verbatim}
	\usepackage{color}
	\usepackage{algorithm}      % pseudo-code
	\usepackage{algpseudocode}  %
	\usepackage{stmaryrd}       % double brackets
	\usepackage{amstext}    % \text macro
	\usepackage{array}      % \newcolumntype macro
	\usepackage{tikz}       % for flow chart
	\usepackage{afterpage}
	\usepackage[export]{adjustbox}
	\usepackage{tensor}
	\usepackage{braket}
	%   \usepackage{commath}    % for abs and norm
	
	\setcounter{secnumdepth}{-2} % remove section numbering

	\makeatletter
	\renewcommand\subparagraph{\@startsection{subparagraph}{5}{\parindent}%
		{3.25ex \@plus1ex \@minus .2ex}%
		{0.75ex plus 0.1ex}% space after heading
		{\normalfont\normalsize\bfseries}}
	\makeatother
	
	\newcommand\independent{\protect\mathpalette{\protect\independenT}{\perp}}
	\def\independenT#1#2{\mathrel{\rlap{$#1#2$}\mkern2mu{#1#2}}}
	\newcommand{\rp}{\mathbb{RP}}
	
	%   Sets
	\newcommand{\nat}{\mathbb{N}}
	\newcommand{\inte}{\mathbb{Z}}
	\newcommand{\re}{\mathbb{R}}
	\newcommand{\renn}{\mathbb{R}_0^+}
	\newcommand{\co}{\mathbb{C}}
	\newcommand{\hil}{\mathbb{H}}
	\newcommand{\ee}{\mathrm{e}}
	\newcommand{\dd}{\mathrm{d}}

	\newcommand{\id}{\indices}
	%   \newcommand{\cp}{\mathbb{CP}}
	%   \newcommand{\dS}{\mathbb{S}}
	%   \newcommand{\dP}{\mathbb{P}}
	%   \newcommand{\dE}{\mathbb{E}}
	%   \newcommand{\dZ}{\mathbb{Z}}
	\newcommand{\bfP}{\mathbf{P}}
	\newcommand{\bfJ}{\mathbf{J}}
	\newcommand{\bfK}{\mathbf{K}}
	\newcommand{\bfR}{\mathbf{R}}	
	%   \newcommand{\bm}{\boldsymbol{m}}
	%   \newcommand{\bmu}{\boldsymbol{\mu}}
	%   \newcommand{\bS}{\boldsymbol{\Sigma}}
	%   \newcommand{\uvec}[1]{\mathrm{\mathbf{\hat{e}}}_#1}
	%   \newcommand{\rmbf}[1]{\mathrm{\mathbf{#1}}}
	%   \newcommand{\javg}{J_{\mathrm{avg^2}}}
	%   \newcommand{\pgl}[1]{\mathbf{PGL}(#1,\mathbb{R})}
	%   \newcommand{\Sl}[1]{\mathbf{SL}(#1,\mathbb{R})}
	%   \newcommand{\gl}[1]{\mathbf{GL}(#1,\mathbb{R})}

	\makeatletter
	\newcommand\etc{etc\@ifnextchar.{}{.\@}}
	\newcommand\ie{i.e\@ifnextchar.{}{.\@}}
	\newcommand\Eq{Eq.\ }
	\makeatother
	
	\newcommand{\red}[1]{{\color{red} #1}}
	\newcommand{\blue}[1]{{\color{blue} #1}}		
	
	\newcommand{\power}{\mathcal{P}}
	\newcommand{\domain}{\mathcal{D}}
	

	
	\newcommand{\na}{\nabla}
	\newcommand{\abs}[1]{\left\lvert #1 \right\rvert}
	\newcommand{\card}[1]{\left\lvert #1 \right\rvert}
	\newcommand{\norm}[1]{\left\lVert #1 \right\rVert}
	\newcommand{\gaussian}{\mathcal{N}}
	\newcommand{\define}{\coloneqq}
	\newcommand{\tp}[1]{{#1}^T}
	\newcommand{\hadj}[1]{{#1}^{\dagger}}
	%
	%   \newcommand{\lst}[2]{\{#1_{1}, #1_{2}, \dots, #1_{#2}\}}
	%   \newcommand{\lstf}[2]{\{#1{1}, #1{2}, \dots, #1{#2}\}}
	%   % prt stands for parenthesis
	%   \newcommand{\prt}[2]{(#1_{1}, #1_{2}, \dots, #1_{#2})}
	%   \newcommand{\prtf}[2]{(#1{1}, #1{2}, \dots, #1{#2})}
	%   % general list formatted, #1: fxn, #2: first one, #3: last one, #4: delimiter, #5: left, #6: right
	%   \newcommand{\glstf}[6]{#5 #1{#2} #4 #1{\number\numexpr#2+1\relax} #4 \dots #4 #1{#3} #6}
	%
	% wc = wild card
	\newcommand*{\wcthin}{{\mkern 2mu\cdot\mkern 2mu}}
	\newcommand*{\wc}{{}\cdot{}}    %   This one is wider
	%
	% Operators
	% ec = equivalence class
	\newcommand{\ec}[1]{\left[ {#1} \right]}
	%\newcommand{\ec}[1]{\left\langle {#1} \right\rangle}
	%
	%   automatic math mode in tabular
	\newcolumntype{L}{>{$}l<{$}}
	\newcolumntype{C}{>{$}c<{$}}
	\newcolumntype{R}{>{$}r<{$}}
	
	\newenvironment{centabular}{\center\tabular}{\endtabular\endcenter}
	\newenvironment{centikzpic}{\center\tikzpicture}{\endtikzpicture\endcenter}
	\newenvironment{eqlong}{\equation\aligned}{\endaligned\endequation}
	
	
	\DeclareMathOperator*{\argmin}{arg\,min}
	\DeclareMathOperator*{\argmax}{arg\,max}
	\DeclareMathOperator{\Var}{Var}
	\DeclareMathOperator{\Cov}{Cov}
	\DeclareMathOperator{\rank}{rank}
	\DeclareMathOperator{\spn}{span}
	\DeclareMathOperator{\diag}{diag}
	\DeclareMathOperator{\tr}{tr}
	
	\newtheorem*{prop*}{Proposition}
	\newtheorem{prop}{Proposition}[section]
	\newtheorem*{lem*}{Lemma}
	\newtheorem{lem}[prop]{Lemma}
	\newtheorem{cor}[prop]{Corollary}
	\newtheorem{thm}[prop]{Theorem}
	\newtheorem*{thm*}{Theorem}
	\newtheorem{conj}[prop]{Conjecture}
	
	\theoremstyle{definition}
	\newtheorem*{def*}{Definition}
	
	\theoremstyle{remark}
	\newtheorem*{rem*}{Remark}
	\newtheorem*{ack*}{Acknowledgements}
	
	\captionsetup{width=0.9\textwidth}
	
	
	%%  \usetikzlibrary{shadows}% for shadow
	%%  \tikzstyle{event} = [color=black!40,text=white,text centered,circular drop shadow,font=\large\bfseries,text height=4em,text width=4em]
	%   \tikzstyle{event} = [draw, circle]
	%   \tikzstyle{arrow} = [thick,->,>=stealth]
	%%  \usetikzlibrary{arrows}
	%%  \tikzstyle{arrow} = [draw, -latex', thick]
	%
	%   %only for this doc
	%   \newcommand{\llb}{\llbracket}
	%   \newcommand{\rrb}{\rrbracket}
	
%opening
\title{Reading Notes for \\ \large \textit{The Quantum Theory of Fields}}
\author{Zhi Wang}

\begin{document}
	
	\maketitle
	
	\tableofcontents
	
	\section{Notes}	
	
	\red{Red} means question.	\blue{Blue} means important notes.
	
	\section{Volume 1}
	\subsection{Chapter 2: Relativistic Quantum Mechanics}
	\subsubsection{2.2 Symmetries}
	\paragraph{P51 (69)}
	\subparagraph{\Eq (2.2.8)}
	
	For \textit{unitary} and \textit{linear} operators,
	plug \Eq(2.2.6) into \Eq(2.2.2),
	\[\left(U\Phi,U\Psi\right) = \left(\Phi,\hadj{U} U\Psi\right) = \left(\Phi,\Psi\right),\,\, \forall\Phi,\Psi. \]
	
	For \textit{antiunitary} and \textit{antilinear} operators,
	plug \Eq (2.2.7) into \Eq (2.2.4),
	\[\left(U\Phi,U\Psi\right) = \left(\Phi,\hadj{U} U\Psi\right)^{*} = \left(\Phi,\Psi\right)^{*},\,\, \forall\Phi,\Psi. \]
	
	Therefore, $\hadj{U}U = 1$. As \red{symmetric transformation should be invertible}, $\hadj{U}=U^{-1}$.
	
	\paragraph{P54 (72)}
	
	\subparagraph{\Eq (2.2.17)}
	\red{Why $t_{bc} = t_{cb}$? (commutativity of derivatives?)}
	
	\subparagraph{\Eq (2.2.21)}
	
	From \Eq (2.2.20)
	\begin{eqlong}
	&	1 + i\theta\id{^a}t\id{_a} + i\bar{\theta}\id{^a}t\id{_a}
	 + \frac{1}{2}\theta\id{^b}\theta\id{^c}t\id{_{bc}}
     + \frac{1}{2}\bar{\theta}\id{^b}\bar{\theta}\id{^c}t\id{_{bc}}
     - \bar{\theta}\id{^{a'}}t\id{_{a'}}\theta\id{^{a''}}t\id{_{a''}} + \cdots \\
     &= 1 + i\theta\id{^a}t\id{_a} + i\bar{\theta}\id{^a}t\id{_a}
     +i f\id{^a_{bc}}\bar{\theta}\id{^b}\theta\id{^c}t\id{_{a}}\\
     &
     + \frac{1}{2}\theta\id{^b}\theta\id{^c}t\id{_{bc}}
     + \frac{1}{2}\bar{\theta}\id{^b}\bar{\theta}\id{^c}t\id{_{bc}}
     + \frac{1}{2}\bar{\theta}\id{^b}\theta\id{^c}t\id{_{bc}}
     + \frac{1}{2}\theta\id{^b}\bar{\theta}\id{^c}t\id{_{bc}}
     +\cdots.\\
	\end{eqlong}

	Replace $a'$ with $b$ and $a''$ with $c$, and exchange $b$ with $c$ for the last term on the right hand side.
	We can then remove the first five terms on the left hand side and get
	
	\begin{eqlong}
		- \bar{\theta}\id{^{b}} t\id{_{b}} \theta\id{^{c}} t\id{_{c}}
		= i f\id{^a_{bc}}\bar{\theta}\id{^b}\theta\id{^c}t\id{_{a}}
		+ \frac{1}{2}\bar{\theta}\id{^b}\theta\id{^c}t\id{_{bc}}
		+ \frac{1}{2}\theta\id{^c}\bar{\theta}\id{^b}t\id{_{cb}}.
	\end{eqlong}

	Because $t\id{_{bc}} = t\id{_{cb}}$, and because \red{$t\id{_b}$ and $\theta\id{^c}$, $\bar{\theta}\id{^b}$ and $\theta\id{^c}$,
		$\theta$ and $f\id{^a_{bc}}$ are commutative
	(why? because $\theta$ are just real numbers?)},
	we get the \Eq (2.2.21):
	
	\begin{eqlong}
		t\id{_{bc}}=-t\id{_{b}}t\id{_{c}} -i f\id{^a_{bc}}t\id{_{a}}.
	\end{eqlong}
	
	\subparagraph{\Eq (2.2.23)}
	\red{Why is $C\id{^a_{bc}} \equiv -f\id{^a_{bc}} + f\id{^a_{cb}}$ a constant? Why is it nonzero but $t_{bc} = t_{cb}$?
	Aren't derivatives commutative?}

	\paragraph{P55 (73)}
	
	\subparagraph{\Eq (2.2.25)}
	In general the Lie bracket of a connected Lie group is always 0 if and only if the Lie group is \textit{Abelian}.
	(\url{https://en.wikipedia.org/wiki/Lie_group#The_Lie_algebra_associated_with_a_Lie_group})
	
	\subparagraph{\Eq (2.2.26)}
	Plug \Eq(2.2.24) into \Eq(2.2.15) and apply it $N-1$ times ($\theta'=\bar{\theta'}=\frac{\theta}{N}$).
	
	\subsubsection{2.3 Quantum Lorentz Transformations}

	\paragraph{P57 (75)}
	\subparagraph{\Eq (2.3.10)}
	\blue{\[\left(\Lambda^{-1}\right)\id{^{\rho}_{\nu}} = \Lambda\id{_{\nu}^{\rho}}.\]}
	
	\subparagraph{Definition of groups}
	\blue{$U(\Lambda,a)$ forms the inhomogeneous Lorentz group, or the Poincaré group.
	$U(\Lambda,0)$ forms the homogeneous Lorentz group.
    If $\det \Lambda=1$, it is proper.
    If $\Lambda\id{^0_0}\ge 1$, it is orthochronous.}
	
	\subsubsection{2.4 The Poincaré algebra}
	\paragraph{P59 (77)}
	\subparagraph{\Eq (2.4.2)}
	\blue{
		Indices lowered and raised:
		
		\begin{eqlong}
			\omega\id{_{\sigma\rho}} &= \eta\id{_{\mu\sigma}}\omega\id{^{\mu}_{\rho}},\\
			\omega\id{^{\mu}_{\rho}} &= \eta\id{^{\mu\sigma}}\omega\id{_{\sigma\rho}}.\\
		\end{eqlong}
	}

	When $\omega$ is infinitesimal, the matrix $\Lambda$ is expressed as
	\begin{equation}
		\Lambda=
		\begin{pmatrix}
			1 & \omega\id{^0_1}& \omega\id{^0_2}&\omega\id{^0_3}\\
			\omega\id{^0_1} & 1& \omega\id{^1_2}&-\omega\id{^3_1}\\
			\omega\id{^0_2} & -\omega\id{^1_2}& 1&\omega\id{^2_3}\\
			\omega\id{^0_3} & \omega\id{^3_1}& -\omega\id{^2_3}&1\\
		\end{pmatrix}.
	\end{equation}

	\subparagraph{\Eq(2.4.5)}
	\red{This is because $J$ is just the derivative of $U$ over $\omega$?}

	\paragraph{P60 (78)}
	\subparagraph{\Eq (2.4.8)}
	
	\begin{eqlong}
		\left(\Lambda\omega\Lambda^{-1}\right)\id{_{\mu\nu}}
		&= \eta\id{_{\kappa\mu}}\Lambda\id{^\kappa_\tau}
		\left(\eta\id{^{\tau\rho}}
		\eta\id{_{\tau\rho}}\right)
		\omega\id{^\tau_\sigma}\left(\Lambda^{-1}\right)\id{^\sigma_{\nu}}\\
		&=\left(\eta\id{_{\kappa\mu}}\Lambda\id{^\kappa_\tau}
		\eta\id{^{\tau\rho}}\right)\left(
		\eta\id{_{\tau\rho}}
		\omega\id{^\tau_\sigma}\right) \left(\Lambda^{-1}\right)\id{^\sigma_{\nu}}\\
		&=\Lambda\id{_\mu^\rho}
		\omega\id{_\rho_\sigma}\Lambda\id{_\nu^{\sigma}},\\
	\end{eqlong}

	Using \Eq (2.4.2),
	\begin{eqlong}
		\left(\Lambda\omega\Lambda^{-1}a\right)\id{_{\mu}}P\id{^\mu}
		&=\Lambda\id{_\mu^\rho}
		\omega\id{_\rho_\sigma}\Lambda\id{_\nu^{\sigma}}a\id{^\nu}P\id{^\mu}\\
		&=\Lambda\id{_\nu^\sigma}
		\omega\id{_\sigma_\rho}\Lambda\id{_\mu^{\rho}}a\id{^\mu}P\id{^\nu}\\
		&=-\Lambda\id{_\nu^\sigma}
		\omega\id{_\rho_\sigma}\Lambda\id{_\mu^{\rho}}a\id{^\mu}P\id{^\nu}.\\
	\end{eqlong}

	
	\subparagraph{\Eq(2.4.9)}
	\[\left(\Lambda\epsilon\right)\id{_\mu}P\id{^\mu} = \Lambda\id{_\mu^\rho}\epsilon\id{_\rho}P\id{^\mu}. \]
	
	For pure 
		translations (with $\Lambda\id{^\mu_\nu} = \delta\id{^\mu_\nu}$),
		they tell us that $P\id{^\rho}$ is translation-invariant, 
		..., the change of the space-space components 
		of $J\id{^{\rho\sigma}}$ under a spatial translation is \red{just the usual change of the angular 
		momentum under a change of the origm relative to which the angular 
		momentum is calculated}.
	
	\subparagraph{\Eq(2.4.10)}
	
	\begin{eqlong}
		\Lambda\id{_\mu^\nu} & = \delta\id{_\mu^\nu} + \omega\id{_\mu^\nu}, \\
		U &= 1 + i \left(\frac{1}{2}\omega\id{_{\mu\nu}}J\id{^{\mu\nu}}-\epsilon\id{_\rho}P\id{^\rho}\right)+\cdots,\\
		U^{-1} &= 1 - i \left(\frac{1}{2}\omega\id{_{\mu\nu}}J\id{^{\mu\nu}}-\epsilon\id{_\rho}P\id{^\rho}\right) + \cdots,\\
		U J\id{^{\rho\sigma}} U^{-1} &= J\id{^{\rho\sigma}}\\
		&+ i \left(\frac{1}{2}\omega\id{_{\mu\nu}}J\id{^{\mu\nu}}-\epsilon\id{_\rho}P\id{^\rho}\right)J\id{^{\rho\sigma}}
		- i \left(\frac{1}{2}\omega\id{_{\mu\nu}}J\id{^{\mu\nu}}-\epsilon\id{_\rho}P\id{^\rho}\right) J\id{^{\rho\sigma}}\\
		&+ \cdots,\\
		\mathrm{rhs} &= \left(\delta\id{_\mu^\rho} + \omega\id{_\mu^\rho}\right)
		\left(\delta\id{_\nu^\sigma} + \omega\id{_\nu^\sigma}\right)
		\left(J\id{^{\mu\nu}}
		-\epsilon\id{^\mu}P\id{^\nu}
		+\epsilon\id{^\nu}P\id{^\mu}\right)\\
		&= \delta\id{_\mu^\rho}\delta\id{_\nu^\sigma}J\id{^{\mu\nu}}
		+\omega\id{_\mu^\rho}\delta\id{_\nu^\sigma}J\id{^{\mu\nu}}
		+\delta\id{_\mu^\rho} \omega\id{_\nu^\sigma}J\id{^{\mu\nu}}\\
		&-\delta\id{_\mu^\rho}\delta\id{_\nu^\sigma}\epsilon\id{^\mu}P\id{^\nu}
		+\delta\id{_\mu^\rho}\delta\id{_\nu^\sigma}\epsilon\id{^\nu}P\id{^\mu}
		+\cdots\\
		&= J\id{^{\rho\sigma}}
		+\omega\id{_\mu^\rho}J\id{^{\mu\sigma}}
		+\omega\id{_\nu^\sigma}J\id{^{\rho\nu}}	
		-\epsilon\id{^\rho}P\id{^\sigma}
		+\epsilon\id{^\sigma}P\id{^\rho}
		+\cdots.\\	
	\end{eqlong}

	\subparagraph{\Eq(2.4.12)}
	Switching $\mu$ with $\nu$ and applying \Eq(2.4.2):
	\[\omega\id{_\mu_\nu}\eta\id{^{\nu\rho}}J\id{^{\mu\sigma}}=\omega\id{_\nu_\mu}\eta\id{^{\mu\rho}}J\id{^{\nu\sigma}}=-\omega\id{_\mu_\nu}\eta\id{^{\mu\rho}}J\id{^{\nu\sigma}}\]
	
	\paragraph{P61 (79)}
	\subparagraph{\Eq(2.4.16)}
	\red{
	Why is $P\id{^0}$ the Hamiltonian $H$?
	Why do $\bfP, \bfJ$ have those physical meanings?
	Why do they conserve?
	Why is $\bfK$ not conserved?
	Why do you use eigenvalues of conserved operator to label physical states?
	}

	\subparagraph{\Eq(2.4.18)}
	
	In coordinate representation,
	
	\begin{eqlong}
		R_i & = x\id{^i},\\
		P_i & = -i\hbar \frac{\partial}{\partial x\id{^i}}.\\
	\end{eqlong}

    Therefore for any $\psi \in \hil$,
    
    \begin{eqlong}
    	\left[R_i, P_j\right]\psi=-x\id{^i}i\hbar \frac{\partial\psi}{\partial x\id{^j}}
    	+i\hbar \frac{\partial\left(x\id{^i}\psi\right)}{\partial x\id{^j}}
    	=\delta_{ij}i\hbar\psi.
    \end{eqlong}

    By the definition of the angular momentum,
    
    \begin{eqlong}
    	J_i = \epsilon_{ijk} R_j P_k.
    \end{eqlong}

	From $\left[R_i,R_j\right]=\left[P_i,P_j\right]=0$, we have,
	
	\begin{eqlong}
		\left[J_i,J_j\right]&=\epsilon_{iab} R_a P_b \epsilon_{jcd} R_c P_d-\epsilon_{jcd} R_c P_d \epsilon_{iab} R_a P_b\\
		&=\epsilon_{iab}\epsilon_{jcd} \left(R_a\left(R_cP_b-\left[R_c,P_b\right]\right)P_d
		                                    -R_c\left(R_aP_d-\left[R_a,P_d\right]\right)P_b\right)\\
		&=i\hbar\epsilon_{iab}\epsilon_{jcd} \left(-R_a \delta_{bc}P_d
				+R_c\delta_{da}P_b\right).\\
	\end{eqlong}
    
    For the first term, $b=c$, so it must be different from $i, j$, let it be $k$.
    Then replace $d$ with $b$.
    
    For the second term, $d=a$, so it must be different from $i, j$, let it be $k$.
    Then replace $c$ with $a$.
    
    \[\left[J_i,J_j\right]=i\hbar\left(-\epsilon_{iak}\epsilon_{jkb}R_aP_b
    +\epsilon_{ikb}\epsilon_{jak}R_aP_b\right).\]

	The terms $i=j,a=b$ cancels out, so $i=b,j=a$ for the first term and $i=a,j=b$ for the second term.
	
	\begin{eqlong}
		\left[J_i,J_j\right]&=i\hbar\left(-\epsilon_{ijk}\epsilon_{jki}R_jP_i
		+\epsilon_{ikj}\epsilon_{jik}R_iP_j\right)\\
		&=i\hbar\left(\epsilon_{ijk}\epsilon_{kji}R_jP_i
		+\epsilon_{ijk}\epsilon_{kij}R_iP_j\right)\\
		&=i\hbar\epsilon_{ijk}\epsilon_{kab}R_aP_b\\
		&=i\hbar\epsilon_{ijk}J_k.\\
	\end{eqlong}

	\subparagraph{\Eq(2.4.21)}
	
	\begin{eqlong}
		\left[J_i,P_j\right]&=\epsilon_{iak}R_aP_kP_j-P_j\epsilon_{iak}R_aP_k\\
		&=\epsilon_{iak}R_aP_jP_k-\epsilon_{iak}P_jR_aP_k\\
		&=\epsilon_{iak}\left[R_a,P_j\right]P_k\\
		&=\epsilon_{iak}\delta_{aj}i\hbar P_k\\
		&=i\hbar\epsilon_{ijk}P_k.\\
	\end{eqlong}

	\subparagraph{\Eq(2.4.27)}
	
	\begin{eqlong}
		J_i&=\epsilon_{ijk} J\id{^{jk}},\\
		\theta^i&=\frac{1}{2}\epsilon^{ijk}\omega\id{_{jk}},\\
	\end{eqlong}

    \ie, 
    
	\begin{eqlong}
		\bfJ &= \left\{J\id{^{23}},J\id{^{31}},J\id{^{12}}\right\}, \\
		\boldsymbol{\theta} &=
		 \left\{ \omega\id{_{23}}, \omega\id{_{31}}, \omega\id{_{12}} \right\}.\\
	\end{eqlong}

	\red{Why is there a negative sign difference from the standard definition?}
	\url{https://en.wikipedia.org/wiki/Rotation_operator_(quantum_mechanics)#Quantum_mechanical_rotations}

	\paragraph{P62 (80)}
	
	\subparagraph{\Eq(2.4.29)}
	
	Given two Hermitian operators $A$ and $B$, and two real-valued parameters $x$ and $y$,
	construct two unitary operators
	\[U=\ee^{iAx}, V=\ee^{iBy}.\]
	Then, up to second order:
	\begin{eqlong}
		U&=1+iAx-\frac{1}{2}A^2x^2+\cdots,\\
		V&=1+iBy-\frac{1}{2}B^2y^2+\cdots,\\
		\ee^{iAx+iBy}&=1+iAx+iBy-\frac{1}{2}A^2x^2-\frac{1}{2}B^2y^2-\frac{1}{2}AxBy-\frac{1}{2}ByAx+\cdots,\\
		UV &= 1+iAx+iBy-\frac{1}{2}A^2x^2-\frac{1}{2}B^2y^2-AxBy+\cdots,\\
	\end{eqlong}

	Therefore
	\begin{eqlong}
		UV &= \left(1-\frac{1}{2}AxBy+\frac{1}{2}ByAx+\cdots\right)\ee^{iAx+iBy}\\
		&= \ee^{-\frac{1}{2}\left[A,B\right]xy}\ee^{iAx+iBy}.\\
	\end{eqlong}

	In this equation, $A=K_i$, $B=-P_j$,
	therefore $-\frac{1}{2}\left[A,B\right]=\frac{1}{2}\left[K_i,P_j\right]=iM\delta_{ij}/2$,
	\[\ee^{i\bfK\cdot\mathbf{v}} \ee^{-i\bfP\cdot\mathbf{a}} = \ee^{iM\delta_{ij}v_i a_j/2 } \ee^{i\left(\bfK\cdot\mathbf{v}-\bfP\cdot\mathbf{a}\right)}
	=\ee^{iM\mathbf{a}\cdot\mathbf{v}/2 } \ee^{i\left(\bfK\cdot\mathbf{v}-\bfP\cdot\mathbf{a}\right)}\]
	
	\subparagraph{Conserved quantities}

	\blue{Translocation (space and time) does not change the momentum (and energy). Rotation does not shift the angular momentum.}
	
	For a pure translation $T(1,a)$ in the subgroup of the Poincaré group,
	\begin{eqlong}
		\braket{p|T(1,a)|\psi}&=
		\braket{p|\ee^{-iP\id{^\mu}a\id{_\mu}}|\psi}=\ee^{-i\mathbf{p}\cdot\mathbf{a}}\braket{p|\psi},\\
	\end{eqlong}
	which demonstrates that the translation does not change the momentum (and energy) of a system.
	
	Similarly, rotation does not shift the angular momentum:
	\[\braket{j|\ee^{i\bfJ\cdot\boldsymbol{\theta}}|\psi}=\ee^{i\bfJ\cdot\boldsymbol{\theta}}\braket{j|\psi}.\]
	
	\subsubsection{2.5 One-Particle States}
	
	\paragraph{P64 (82)}
	\subparagraph{\Eq(2.5.4)}
	
	The invariant $p^2$ is equivalent to $-M^2$ in time-like domain, where $M$ is the rest mass.
	
	Consider the four dimensional vector space over $\re$ to which the four-vector momentum $p\id{^\mu}$ belongs.
	The proper orthochronous Lorentz group divides the vector space into equivalent classes:
	\[p\id{^\mu}=L\id{^\mu_\nu}\left(p\right)k\id{^\nu},\]
	\ie, for \textit{each} $p\id{^\mu}$, there \textit{exists a unique} image $k\id{^\nu}$ in the quotient space. The mapping is \textit{represented uniquely} by $L\left(p\right)$.
	For \textit{each} element $k\id{^\nu}$ in the quotient space, there \textit{exists a unique} equivalent class containing
	four-vector momenta whose mapping towards the quotient space is $k\id{^\nu}$.
	Momenta falling into (belonging to) different equivalent classes are unreachable (not connected) through proper orthochronous Lorentz transformation.
	
	The proper orthochronous Lorentz group divides the four-vector momentum space into three components:
	the component where the invariant square is positive
	($\left(p^0\right)^2 < \left(p^1\right)^2 + \left(p^2\right)^2 + \left(p^3\right)^2$, the space-like domain),
	the time-like component of positive energy
	($\left(p^0\right)^2 > \left(p^1\right)^2 + \left(p^2\right)^2 + \left(p^3\right)^2$ with $p^0>0$),
	and
	the time-like component of negative energy
		($\left(p^0\right)^2 > \left(p^1\right)^2 + \left(p^2\right)^2 + \left(p^3\right)^2$ with $p^0<0$).
	
	\subparagraph{\Eq(2.5.5)}
	This is to \textit{define} the $\sigma$ states in $\Psi_p$ space, with reference to $\Psi_{k,\sigma}$.
	\blue{The assumption is that the $\sigma$ label is orthogonal to the momenta label,
	and that the state space of $\Psi_k$ has the same size of that of $\Psi_p$},
	\ie, there exists at least one bijection from state space $\{\Psi_{k,\sigma},\forall \sigma\}$ to state space $\{\Psi_{p,\sigma'}, \forall \sigma'\}$.
	
	This is to choose the way of labeling $\sigma$ such that the $C$ matrix in \Eq(2.5.3) is an identity matrix multiplied by a constant.
	
	\paragraph{P65 (83)}
	\subparagraph{\Eq(2.5.12)}
	\red{Why is it $\delta^3$ instead of $\delta^4$? How about the time (0-th) component of the $k$ vector?}
	\subparagraph{\Eq(2.5.13)}
	Since $U$ is symmetric operation (\Eq(2.2.2)), it does not change the inner product:
	\[\braket{U\Psi_{k,\sigma'}|U\Psi_{k,\sigma}}=\braket{\Psi_{k,\sigma'}|\Psi_{k,\sigma}}=\delta_{\sigma'\sigma}.\]
	
	On the other hand, from \Eq(2.5.8),
	\[U\Psi_{k,\sigma}=D_{\rho\sigma}\Psi_{k,\rho},U\Psi_{k,\sigma'}=D_{\rho'\sigma'}\Psi_{k,\rho'}.\]
	
	Therefore, from \Eq(2.1.2) and \Eq(2.1.3), and plug in \Eq(2.5.12),
	\[\delta_{\sigma'\sigma}=D_{\rho'\sigma'}^{*}D_{\rho\sigma}\braket{\Psi_{k,\rho'}|\Psi_{k,\rho}}=D_{\rho'\sigma'}^{*}D_{\rho\sigma}\delta_{\rho'\rho}=D_{\rho\sigma'}^{*}D_{\rho\sigma},\]
	where the right hand side is just
	\[D_{\rho\sigma'}^{*}D_{\rho\sigma}=\left(\hadj{D}\right)_{\sigma'\rho}D_{\rho\sigma}=\left(\hadj{D}D\right)_{\sigma'\sigma}.\]
	
	Since symmetric operation is invertible, so is $D$.
	Thus, it can be concluded that
	\[\hadj{D}=D^{-1}.\]
	
	\paragraph{P66 (84)}
	\subparagraph{Table 2.1}
	
	The energy $p^0$ is put last.
	
	If $\Lambda$ in \Eq(2.5.6) is in the (homogeneous) Lorentz group $O(3,1)$,
	then the little group for $k\id{^\mu}=0$ should also be $O(3,1)$.
	Maybe all ``S''s in the table should be discarded.
	
	If $\Lambda$ in \Eq(2.5.6) is in the proper orthochronous (homogeneous) Lorentz group $SO^+(3,1)$,
	then the little group for $k\id{^\mu}=0$ should also be $SO^+(3,1)$.

	(e): $SO^+(2,1)$.
	
	For (c) and (d), consider a infinitesimal $1+\dd\Lambda$ matrix:
	\begin{equation}
		1+\dd\Lambda=
		\begin{pmatrix}
			1 & \omega\id{^0_1}& \omega\id{^0_2}&\omega\id{^0_3}\\
			\omega\id{^0_1} & 1& \omega\id{^1_2}&-\omega\id{^3_1}\\
			\omega\id{^0_2} & -\omega\id{^1_2}& 1&\omega\id{^2_3}\\
			\omega\id{^0_3} & \omega\id{^3_1}& -\omega\id{^2_3}&1\\
		\end{pmatrix},
	\end{equation}
	acting on
	\[\begin{pmatrix}
		\kappa\\\kappa\\0\\0\\
	\end{pmatrix},\]
    then
    \[\left(1+\omega\id{^0_1}\right)\kappa=\kappa,
    \left(\omega\id{^0_2} - \omega\id{^1_2}\right)\kappa=0,
    \left(\omega\id{^0_3} + \omega\id{^3_1}\right)\kappa=0.\]
    
    Therefore, $1+\dd\Lambda$ can be rewritten as
    \begin{eqlong}\label{eqdLambda}
    	1+\dd\Lambda=
    	\begin{pmatrix}
    		1 & 0& -a&-b\\
			0 & 1& -a&-b\\
    		-a & a& 1&-\theta\\
    		-b & b& \theta&1\\
    	\end{pmatrix},
    \end{eqlong}
	or
	\begin{eqlong}
		\Lambda&=\ee^{\dd\Lambda}\\
		&=\sum_{i=0}^{+\infty}\frac{\left(\dd\Lambda\right)^i}{i!}\\
		&=
		\begin{pmatrix}
			1+\left(a^2+b^2\right)f(\theta) & -\left(a^2+b^2\right)f(\theta)&  bg(\theta)+ah(\theta)&-ag(\theta)+bh(\theta)\\
			\left(a^2+b^2\right)f(\theta) & 1-\left(a^2+b^2\right)f(\theta)&  bg(\theta)+ah(\theta)&-ag(\theta)+bh(\theta)\\
			-bg(\theta)+ah(\theta) & bg(\theta)-ah(\theta)& \cos\theta&-\sin\theta\\
			ag(\theta)+bh(\theta) & -ag(\theta)-bh(\theta)& \sin\theta&\cos\theta\\
		\end{pmatrix},
	\end{eqlong}
	where
	\begin{eqlong}
		f(\theta) &= \sum_{n=0}^{+\infty}\frac{1}{\left(2n\right)!}\frac{(-1)^n}{(2n+1)(2n+2)}\theta^{2n}\\
		&=\sum_{n=1}^{+\infty}\frac{(-1)^{n+1}}{\left(2n\right)!}\theta^{2n-2},\\
		g(\theta) &= \sum_{n=1}^{+\infty}\frac{(-1)^n}{\left(2n\right)!}\theta^{2n-1},\\
		h(\theta) &= \sum_{n=0}^{+\infty}\frac{(-1)^{n+1}}{\left(2n+1\right)!}\theta^{2n}.\\
	\end{eqlong}
	Since
	\begin{eqlong}
		\cos\theta&=\sum_{n=0}^{+\infty}\frac{(-1)^{n}}{\left(2n\right)!}\theta^{2n},\\
		\sin\theta&=\sum_{n=0}^{+\infty}\frac{(-1)^{n}}{\left(2n+1\right)!}\theta^{2n+1},\\
	\end{eqlong}
    we have
    \begin{eqlong}
    	f(\theta)&=\frac{1-\cos\theta}{\theta^2},\\
    	g(\theta)&=\frac{\cos\theta-1}{\theta},\\
    	h(\theta)&=-\frac{\sin\theta}{\theta}.\\
    \end{eqlong}

	In conclusion,
	\begin{eqlong}\label{eqLambda}
		\Lambda&=
		\begin{pmatrix}
			1+\frac{\left(a^2+b^2\right)(1-\cos\theta)}{\theta^2} &
			 -\frac{\left(a^2+b^2\right)(1-\cos\theta)}{\theta^2}&
			 - \frac{a\sin\theta+b\left(1-\cos\theta\right)}{\theta}&
			 -\frac{b\sin\theta-a\left(1-\cos\theta\right)}{\theta}\\
			\frac{\left(a^2+b^2\right)(1-\cos\theta)}{\theta^2} &
			1-\frac{\left(a^2+b^2\right)(1-\cos\theta)}{\theta^2}& 
			- \frac{a\sin\theta+b\left(1-\cos\theta\right)}{\theta}&
			-\frac{b\sin\theta-a\left(1-\cos\theta\right)}{\theta}\\
			-\frac{a\sin\theta-b\left(1-\cos\theta\right)}{\theta} &
			 \frac{a\sin\theta-b\left(1-\cos\theta\right)}{\theta}&
			  \cos\theta&
			  -\sin\theta\\
			- \frac{b\sin\theta+a\left(1-\cos\theta\right)}{\theta} &
			 \frac{b\sin\theta+a\left(1-\cos\theta\right)}{\theta}& 
			 \sin\theta&
			 \cos\theta\\
		\end{pmatrix},
	\end{eqlong}
	which, when $\theta\to0$, becomes
	\begin{eqlong}\label{eqSmunu}
		\lim\limits_{\theta\to0}\Lambda=
		\begin{pmatrix}
			1+\frac{a^2+b^2}{2} &
			-\frac{a^2+b^2}{2}&
			-a&
			-b\\
			\frac{a^2+b^2}{2} &
			1-\frac{a^2+b^2}{2}& 
			-a&
			-b\\
			-a&
			a&
			1&
			0\\
			-b&
			b&
			0&
			1\\
		\end{pmatrix}.
	\end{eqlong}

	Easy to see
	\[\Lambda(a,b,0)\Lambda(a',b',0)=\Lambda(a+a',b+b',0).\]
	
	$t-x$ and $y-z$ is conserved under $\Lambda(a,b,0)$.
	
	\subparagraph{\Eq(2.5.14)}
	
	\begin{eqlong}
		\braket{\Psi_{p',\sigma'}|\Psi_{p,\sigma}}&=\braket{\Psi_{p',\sigma'}|N(p)U(L(p))\Psi_{k,\sigma}}&\,\,\text{\Eq (2.5.5)}\\
		&=N(p)\braket{\Psi_{p',\sigma'}|U(L(p))\Psi_{k,\sigma}}&\,\,\text{\Eq (2.1.2)}\\
		&=N(p)\braket{\hadj{U(L(p))}\Psi_{p',\sigma'}|\Psi_{k,\sigma}}&\,\,\text{\Eq (2.2.6)}\\
		&=N(p)\braket{U^{-1}(L(p))\Psi_{p',\sigma'}|\Psi_{k,\sigma}},&\,\,\text{\Eq (2.2.8)}\\
	\end{eqlong}
	\begin{eqlong}
		\braket{\Psi_{p',\sigma'}|\Psi_{p,\sigma}}
		&=N(p)\braket{U(L^{-1}(p))\Psi_{p',\sigma'}|\Psi_{k,\sigma}}&\,\,\text{\Eq (2.3.12)}\\
		&=N(p)\braket{\frac{N(p')}{N(k')}\sum_{\sigma''}D_{\sigma''\sigma'}(W(L^{-1}(p),p'))\Psi_{k',\sigma''}|\Psi_{k,\sigma}}&\,\,\text{\Eq (2.5.11)}\\
		&=\frac{N(p)N^*(p')}{N^*(k')}\sum_{\sigma''}D^{*}_{\sigma''\sigma'}(W(L^{-1}(p),p'))\braket{\Psi_{k',\sigma''}|\Psi_{k,\sigma}}&\,\,\text{\Eq (2.1.3)}\\
		&=\frac{N(p)N^*(p')}{N^*(k')}D^{*}_{\sigma\sigma'}(W(L^{-1}(p),p'))\delta^3(\mathbf{k'}-\mathbf{k}),&\,\,\text{\Eq (2.5.12)}\\
	\end{eqlong}
	where $N(k')=1$ due to the definition in \Eq(2.5.5).
	
	\paragraph{P68 (86)}
	\subparagraph{\Eq(2.5.20)}
	\red{How can you find all representations of SO(3)?}
	\url{https://math.stackexchange.com/questions/263313/finding-all-irreducible-representations-of-so3}
	
	
	\begin{thebibliography}{9}
		\bibitem{Representation Theory}
		Joe Harris, and William Fulton.
		\textit{Representation Theory: A First Course}.
		
		\bibitem{Quantum Theory}
		Peter Woit
		\textit{Quantum Theory, Groups and Representations: An Introduction}.
		
	\end{thebibliography}

	
	\subparagraph{\Eq(2.5.23)}
	\blue{$D^{(j)}(R)$ is the $j$-th irreducible representation of $R\in \mathrm{SO}(3)$.
		
		``A particle is spin $j$'' = ``the wave function describing the particle is the $j$-th irreducible representation''.}
	
	\red{It is possible that the wave function is not reducible representation?
		Then it is not single-particle anymore?}
	
	\paragraph{P69 (87)}
	\subparagraph{Mass Zero}
	$W$ should preserve inner product. It should preserve $k$ as well:
	\[(Wt)\id{^\mu}k\id{_\mu}=(Wt)\id{^\mu}(Wk)\id{_\mu}=t\id{^\mu}k\id{_\mu}=-1. \]
	\paragraph{P70 (88)}
	\subparagraph{\Eq(2.5.26)}
	See \eqref{eqSmunu}.
	Exchange 1st and 4th row, 2nd and 3rd row.
	Then exchange 1st/4th column, 2nd/3rd column.
	$a\to-\beta, b\to-\alpha$.
	
	\subparagraph{\Eq(2.5.28)}
	This is inconsistent with \eqref{eqLambda}.
	Because $S(\alpha,\beta)$ is not commutative with $R(\theta)$.
	So $\exp(\dd S+\dd R)\ne \exp(\dd S)\exp(\dd R)$.
	
	\eqref{eqLambda} is a Lorentz transform.
	And it preserves inner product between $k$ and $t$.
	
	To get \eqref{eqLambda}, we need to calculate
	\[S\left(
	-\frac{b\sin\theta+a\left(1-\cos\theta\right)}{\theta},
	 - \frac{a\sin\theta-b\left(1-\cos\theta\right)}{\theta}\right)R\left(\theta\right).\]
	 
	\subparagraph{\Eq(2.5.29)--(2.5.31)}
	Confirmed with Mathematica.
	
	\Eq(2.5.31) shows that the conjugacy class
	$\ec{S(\alpha,\beta)}\subseteq \set{S(\alpha',\beta')|\alpha'\in \re,\beta'\in\re}$
	(because all $S$'s commute).
	
	Actually the conjugacy class is
	\[\ec{S(\alpha,\beta)}= \set{S(\alpha',\beta')|\alpha'^2+\beta'^2=\alpha^2+\beta^2}.\]
	
	\red{Test this is isomorphic to ISO(2).}
	
	\paragraph{P71 (89)}
	\subparagraph{\Eq(2.5.32)}
	Consistent with \eqref{eqdLambda}.
	\subparagraph{Diagonalize}
	\begin{prop*}
		If $A$ and $B$ are Hermitian and $[A,B]=0$, they can be diagonalized simultaneously.
	\end{prop*}
	\begin{proof}
		Since $A$ is Hermitian, there exists
		$S^{-1}AS=D$, where $D$ is diagonal.
		Then the commutativity is expressed as
		\[SDS^{-1}B=BSDS^{-1}.\]
		Left multiply $S^{-1}$ and right multiply $S$:
		\[DS^{-1}BS=S^{-1}BSD.\]
		Let $B'=S^{-1}BS$, then $[D,B']=0$.
		
		To write it in a scalar form:
		\[(B')_{ij}d_i=(B')_{ij}d_j,\]
		where $d$ are diagonal elements in $D$, \ie, eigenvalues of $A$.
		
		For this to be true,
		$B'$ must be block diagonal where the block corresponds to same eigenvalues in $D$,
		in which case there is always a way to adjust $S$ matrix (re-diagonalization) such that $B'$ is diagonal.
	\end{proof}

	\paragraph{P72 (90)}	
	\subparagraph{Masslesss Particles}
	\blue{Massless particles are not observed to have any continuous degree of 
	freedom like $\theta$.}
	\subparagraph{\Eq(2.5.39)}
	Note that $\Psi_{k,\sigma}$ is actually
	$\Psi_{k,a=0,b=0,\sigma}$.
	The $\sigma$ states are diagonalized \textit{within} $a=0,b=0$ states.
	Therefore, this does not contradict with the fact that $A, B$ and $J$ does not commute.
	
	And hence 
	\begin{eqlong}\label{eqUSabPsi}
		U(S(\alpha,\beta))\Psi_{k,\sigma}= \Psi_{k,\sigma}.
	\end{eqlong}
	
	\subparagraph{Helicity}
	\blue{$J_3$ is the direction of motion because the four-vector momentum
		$k$ is chosen to be nonzero in the third spatial axis.}
	
	\paragraph{P73 (91)}
	\subparagraph{Lorentz-invariant Helicity}
	A general Lorentz transform can be decomposed as $L(p)W(\alpha,\beta,\theta)$,
	the former of which has 3 DOF, the latter has 3.
	
	We have shown from \eqref{eqUSabPsi} that $W$ preserves helicity ($R$ preserves as well because it is eigenstate).
	
	In short, this is shown in \Eq(2.5.42), $\Lambda$ should be any proper orthochronous Lorentz transform.
	(The equation holds because $D$ matrix is diagonal.)
	The pure translation preserves $\sigma$ as well because the state is in $p$-eigenstate.
	So does space/time inversion.
	
	\subsubsection{2.6 Space Inversion and Time-Reversal}
	\paragraph{P75 (93)}
	\subparagraph{\Eq(2.6.1)}
	\red{These equations do not hold because there is no connected path for space/time inversion?}
	\paragraph{P76 (94)}
	\subparagraph{\Eq(2.6.7)}
	\blue{Space inversion is linear and time inversion is antilinear. Both commutes with Hamiltonian.}
	\section{Volume 2}
	\section{Volume 3}
	
\end{document}
	