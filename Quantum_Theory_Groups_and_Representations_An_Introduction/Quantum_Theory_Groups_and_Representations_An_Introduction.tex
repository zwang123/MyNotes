\documentclass[12pt, letterpaper]{article}
\usepackage{amsmath,amssymb,amsthm,amsopn,amscd}
\usepackage{mathtools}
\usepackage{latexsym}
\usepackage{graphicx,caption,subcaption}
\usepackage{multirow}
\usepackage[reftex]{theoremref}
\usepackage{hyperref}
\usepackage{verbatim}
\usepackage{color}
\usepackage{algorithm}      % pseudo-code
\usepackage{algpseudocode}  %
\usepackage{stmaryrd}       % double brackets
\usepackage{amstext}    % \text macro
\usepackage{array}      % \newcolumntype macro
\usepackage{tikz}       % for flow chart
\usetikzlibrary{cd}     % commutative diagram
\usetikzlibrary{shapes.geometric} % pentagon
\usepackage{graphics, tkz-berge} % icosahedron
\usepackage{afterpage}
\usepackage[export]{adjustbox}
\usepackage{tensor}
\usepackage{braket}
\usepackage{etoolbox}
\usepackage{xparse}
%   \usepackage{commath}    % for abs and norm

%\setcounter{secnumdepth}{-2} % remove section numbering
\setcounter{section}{-1}

\makeatletter
\renewcommand\subparagraph{\@startsection{subparagraph}{5}{\parindent}%
	{3.25ex \@plus1ex \@minus .2ex}%
	{0.75ex plus 0.1ex}% space after heading
	{\normalfont\normalsize\bfseries}}
\makeatother

\newcommand\independent{\protect\mathpalette{\protect\independenT}{\perp}}
\def\independenT#1#2{\mathrel{\rlap{$#1#2$}\mkern2mu{#1#2}}}
\newcommand{\rp}{\mathbb{RP}}

%   Sets
\newcommand{\nat}{\mathbb{N}}
\newcommand{\inte}{\mathbb{Z}}
\newcommand{\rat}{\mathbb{Q}}
\newcommand{\re}{\mathbb{R}}
\newcommand{\renn}{\mathbb{R}_0^+}
\newcommand{\co}{\mathbb{C}}
\newcommand{\hil}{\mathbb{H}}
\newcommand{\ee}{\mathrm{e}}
\newcommand{\dd}{\mathrm{d}}
\newcommand{\GL}{\mathrm{GL}}
\newcommand{\MM}{\mathrm{M}}
\newcommand{\ob}{\mathrm{ob}}
\newcommand{\dom}{\mathrm{dom}}
\newcommand{\cod}{\mathrm{cod}}
\newcommand{\Hom}{\mathrm{Hom}}
\newcommand{\End}{\mathrm{End}}
\newcommand{\class}{\mathrm{class}}
\newcommand{\supp}{\mathrm{supp}}

\newcommand{\ext}[1]{\bigwedge\!^{#1}}


\newcommand{\id}{\indices}
\newcommand{\idt}{\mathrm{id}}
%   \newcommand{\cp}{\mathbb{CP}}
%   \newcommand{\dS}{\mathbb{S}}
%   \newcommand{\dP}{\mathbb{P}}
%   \newcommand{\dE}{\mathbb{E}}
%   \newcommand{\dZ}{\mathbb{Z}}
\newcommand{\bfP}{\mathbf{P}}
\newcommand{\bfJ}{\mathbf{J}}
\newcommand{\bfK}{\mathbf{K}}
\newcommand{\bfR}{\mathbf{R}}
\newcommand{\idm}{\mathbf{I}}
\newcommand{\bfA}{\mathbf{A}}
\newcommand{\bfB}{\mathbf{B}}
\newcommand{\bfC}{\mathbf{C}}
\newcommand{\bfD}{\mathbf{D}}
\newcommand{\bfG}{\mathbf{G}}
\newcommand{\bfL}{\mathbf{L}}
\newcommand{\bfT}{\mathbf{T}}
\newcommand{\bfS}{\mathbf{S}}
%   \newcommand{\bm}{\boldsymbol{m}}
%   \newcommand{\bmu}{\boldsymbol{\mu}}
%   \newcommand{\bS}{\boldsymbol{\Sigma}}
%   \newcommand{\uvec}[1]{\mathrm{\mathbf{\hat{e}}}_#1}
%   \newcommand{\rmbf}[1]{\mathrm{\mathbf{#1}}}
%   \newcommand{\javg}{J_{\mathrm{avg^2}}}
%   \newcommand{\pgl}[1]{\mathbf{PGL}(#1,\mathbb{R})}
%   \newcommand{\Sl}[1]{\mathbf{SL}(#1,\mathbb{R})}
%   \newcommand{\gl}[1]{\mathbf{GL}(#1,\mathbb{R})}

\makeatletter
\newcommand\etc{etc\@ifnextchar.{}{.\@}}
\newcommand\ie{i.e\@ifnextchar.{}{.\@}}
\newcommand\eg{e.g\@ifnextchar.{}{.\@}}
\newcommand\Eq{Eq.\ }
\makeatother



\newcommand{\red}[1]{{\color{red} #1}}
\newcommand{\blue}[1]{{\color{blue} #1}}		

\newcommand{\power}{\mathcal{P}}
\newcommand{\domain}{\mathcal{D}}



\newcommand{\na}{\nabla}
\newcommand{\abs}[1]{\left\lvert #1 \right\rvert}
\newcommand{\card}[1]{\left\lvert #1 \right\rvert}
\newcommand{\norm}[1]{\left\lVert #1 \right\rVert}
\newcommand{\gaussian}{\mathcal{N}}
\newcommand{\define}{\coloneqq}
\newcommand{\tp}[1]{{#1}^T}
\newcommand{\hadj}[1]{{#1}^{\dagger}}
\newcommand{\conj}{\overline}
%
%   \newcommand{\lst}[2]{\{#1_{1}, #1_{2}, \dots, #1_{#2}\}}
%   \newcommand{\lstf}[2]{\{#1{1}, #1{2}, \dots, #1{#2}\}}
%   % prt stands for parenthesis
%   \newcommand{\prt}[2]{(#1_{1}, #1_{2}, \dots, #1_{#2})}
%   \newcommand{\prtf}[2]{(#1{1}, #1{2}, \dots, #1{#2})}
%   % general list formatted, #1: fxn, #2: first one, #3: last one, #4: delimiter, #5: left, #6: right
%   \newcommand{\glstf}[6]{#5 #1{#2} #4 #1{\number\numexpr#2+1\relax} #4 \dots #4 #1{#3} #6}
%
% wc = wild card
\newcommand*{\wcthin}{{\mkern 2mu\cdot\mkern 2mu}}
\newcommand*{\wc}{{}\cdot{}}    %   This one is wider
%
% Operators
% ec = equivalence class
\newcommand{\ec}[1]{\left[{#1}\right]}
% generating subgroup
\newcommand{\gensub}[1]{\left\langle{#1}\right\rangle}
%
%   automatic math mode in tabular
\newcolumntype{L}{>{$}l<{$}}
\newcolumntype{C}{>{$}c<{$}}
\newcolumntype{R}{>{$}r<{$}}

\newenvironment{centabular}{\center\tabular}{\endtabular\endcenter}
\newenvironment{centikzpic}{\center\tikzpicture}{\endtikzpicture\endcenter}
\newenvironment{eqlong}{\equation\aligned}{\endaligned\endequation}


\DeclareMathOperator*{\argmin}{arg\,min}
\DeclareMathOperator*{\argmax}{arg\,max}
\DeclareMathOperator{\Var}{Var}
\DeclareMathOperator{\Cov}{Cov}
\DeclareMathOperator{\rank}{rank}
\DeclareMathOperator{\spn}{span}
\DeclareMathOperator{\diag}{diag}
\DeclareMathOperator{\tr}{tr}

\newtheorem*{prop*}{Proposition}
\newtheorem{prop}{Proposition}[section]
\newtheorem*{lem*}{Lemma}
\newtheorem{lem}[prop]{Lemma}
\newtheorem{cor}[prop]{Corollary}
\newtheorem{thm}[prop]{Theorem}
\newtheorem*{thm*}{Theorem}
\newtheorem{conjec}[prop]{Conjecture}

%https://tex.stackexchange.com/questions/280313/how-to-put-the-list-of-definitions-at-contents-page
%https://tex.stackexchange.com/questions/51691/creating-list-of-for-newtheoremstyle
%\usepackage{amsthm}
%\newtheoremstyle{mystyle}
%{\topsep}{\topsep}{}{}{\bfseries}{:}{\newline}
%{\thmname{#1}\thmnumber{ #2}\thmnote{ (#3)}%
%	\ifstrempty{#3}%
%	{\addcontentsline{def}{subsection}{#1~\thedef}}%
%	{\addcontentsline{def}{subsection}{#1~\thedef~(#3)}}}
%
%\theoremstyle{mystyle}
%\newtheorem*{def*}{Definition}
\theoremstyle{definition}
\newtheorem*{defaux}{Definition}

%https://tex.stackexchange.com/questions/60872/ams-theorems-in-table-of-contents
\NewDocumentEnvironment{def*}{o}
{\IfNoValueTF{#1}
	{\defaux\addcontentsline{toc}{subsubsection}{\protect\numberline{}Definition}}
	{\defaux[#1]\addcontentsline{toc}{subsubsection}{\protect\numberline{}Definition (#1)}}%
	\ignorespaces}
{\label{#1}}
{\enddefaux}

%\makeatletter
%\newcommand\definitionname{Definition}
%\newcommand\listdefinitionname{List of Definitions}
%\newcommand\listofdefinitions{%
%	\section*{\listdefinitionname}\@starttoc{def}}
%\makeatother

\theoremstyle{remark}
\newtheorem*{rem*}{Remark}
\newtheorem*{ack*}{Acknowledgements}

\theoremstyle{definition}
\newtheorem{exe}{Exercise}[section]
\newtheorem{exe*}[exe]{Exercise*}
\newtheorem{exam}[exe]{Example in Book}
\newtheorem{eq}[exe]{Equation in Book}
\theoremstyle{plain}
\newtheorem{pprop}[exe]{Proposition in Book}
\newtheorem{ccor}[exe]{Corollary in Book}
\newtheorem{llem}[exe]{Lemma in Book}
\newtheorem{tthm}[exe]{Theorem in Book}
\captionsetup{width=0.9\textwidth}


%%  \usetikzlibrary{shadows}% for shadow
%%  \tikzstyle{event} = [color=black!40,text=white,text centered,circular drop shadow,font=\large\bfseries,text height=4em,text width=4em]
%   \tikzstyle{event} = [draw, circle]
%   \tikzstyle{arrow} = [thick,->,>=stealth]
%%  \usetikzlibrary{arrows}
%%  \tikzstyle{arrow} = [draw, -latex', thick]
%
%   %only for this doc
%   \newcommand{\llb}{\llbracket}
%   \newcommand{\rrb}{\rrbracket}

%opening
\title{Reading Notes for \\ \large \textit{Quantum Theory, Groups and Representations:\\An Introduction}}
\author{Zhi Wang}

\begin{document}
	
	\maketitle
	
	\tableofcontents
	
	%	\listofdefinitions
	
	\section{Notes and Definitions}
	\subsection{Notes}
	\section{Introduction and Overview}
	\section{The Group $U(1)$ and its Representations}
	
	\subsection*{2.2 The group $U(1)$ and its representations}
	\paragraph{P19}
	\subparagraph{$\pi_k\colon\co^*\to\co^*$}
	\blue{$k$ must be integer, with the same reason as $U(1)$.}
	If $g,h\in\co^*$ satisfies
	\[g=\ee^{ia}, h=\ee^{ib},\]
	where $a,b\in\re$ and $a+b=2\pi$,
	then $gh=1$. Therefore
	\[\pi_k(g)\pi_k(h)=\ee^{iak}\ee^{ibk}=\ee^{iak+ibk}=\ee^{ik(2\pi)}=\pi_k(1)=1^k=1,\]
	indicating $k\in\inte$.
	
	\subsection*{2.4 Conservation of charge and $U(1)$ symmetry}
	\paragraph{P22}
	\subparagraph{$[U(t),Q]=0$}
	The assumption here is \blue{``if $[H,Q]=0$''}.
	
	\section{Two-state Systems and $SU(2)$}
	\section{Linear Algebra Review, Unitary and Orthogonal Groups}
	\subsection*{4.4 Inner products}
	\paragraph{P38}
	\subparagraph{Definition of Inner Product}
	\begin{def*}[nondegenerate bilinear form]
		A \textbf{nondegenerate bilinear form} is a bilinear form $f\colon V\times V\to K$
		on a vector space $V$ over a field $K$,
		such that $v\mapsto (x\mapsto f(x,v))$ is an isomorphism from $V$ to $V^*$.
	\end{def*}
	\begin{rem*}
		In finite dimensions, this is equivalent to
		\[\forall y \in V [f(x,y)=0] \implies [x=0].\]
	\end{rem*}
	\begin{def*}[sesquilinear form]
		Over a complex vector space $V$ a map $\varphi \colon V\times V\to \co$ is \textbf{sesquilinear} if
		\[\begin{aligned}&\varphi (x+y,z+w)=\varphi (x,z)+\varphi (x,w)+\varphi (y,z)+\varphi (y,w),\\&\varphi (ax,by)={\conj {a}}b\,\varphi (x,y),\end{aligned}\]
		for all $x,y,z,w\in V$ and all $a,b\in \co$. Here, $\conj {a}$ is the complex conjugate of a scalar $a$.
	\end{def*}
	\begin{def*}[Hermitian form]
		A sesquilinear form is \textbf{Hermitian} if and only if $\langle x,x\rangle$ is real for all $x$.
	\end{def*}
	\subsection*{4.7 Eigenvalues and eigenvectors}
	\subsubsection*{Polynomial and Algebraically Closed Field}
	
	\begin{def*}[polynomial ring]
		The \textbf{polynomial ring}, $R[x]$, in $x$ over a commutative ring $R$
		is the set of functions $f\colon R\to R$, called polynomials in $x$, of the form
		\[f(x)=\sum_{i=0}^{n}a_ix^i,\]
		where $a_0, a_1, \dots, a_n$, the \textbf{coefficients} of $f$, are elements of $R$.
	\end{def*}
	\begin{prop}
		If a non-constant polynomial $f(x)\in R[x]$ has a root $\lambda\in R$,
		then $f(x)=(x-\lambda)g(x)$ where $g(x)$ is a nonzero polynomial in $R[x]$.
		The degree of $g(x)$ is one less than that of $f(x)$.
	\end{prop}
	\begin{proof}
		Since $f$ is non-constant with root $\lambda\in R$ of order $n$, it can be expressed by
		\[f(x)=\sum_{i=0}^{n}a_ix^i,\]
		where $n>0$ and $a_n\ne0$.
		Consider $g\colon R\to R$ expressed as (which is undefined if $n=0$)
		\[g(x)=\sum_{i=0}^{n-1}b_ix^i, \]
		where 
		\begin{eqlong}\label{eqCoeffB}
			b_{n-1}&=a_{n}\ne0,\\
			b_k&=a_{k+1}+\lambda b_{k+1},\,\,\,\forall 0\le k<n-1,\\
		\end{eqlong}
		Coefficients $b$'s are in $R$ because all $a$'s and $\lambda$ are in $R$.
		\textbf{Therefore $g(x)\in R[x]$. Its degree is $n-1$,
		and its leading coefficient is nonzero.}
		
		From mathematical induction it can be derived that
		\[b_k = \sum_{m=0}^{n-1-k} a_{k+1+m}\lambda^m. \]
		
		The following shows it satisfies $b_k=a_{k+1}+\lambda b_{k+1}$:
		\[\begin{aligned}
			a_{k+1}+\lambda b_{k+1} &= a_{k+1}+\sum_{m=0}^{n-2-k} a_{k+2+m}\lambda^{m+1}\\
			&=a_{k+1}+\sum_{m'=1}^{n-1-k} a_{k+1+m'}\lambda^{m'}\\
			&=\sum_{m'=0}^{n-1-k} a_{k+1+m'}\lambda^{m'}=b_k,\\
		\end{aligned}\]
		where $m'=m+1$.
		
		Therefore,
		\[b_0\lambda =\sum_{m'=1}^{n}a_{m'}\lambda^{m'},\]
		where $m'=m+1$. Since $\lambda$ is the root of $f(x)$,
		we have
		\begin{eqlong}\label{eqCoeffB0}
			b_0\lambda = -a_0.\\
		\end{eqlong}
		
		In conclusion (in case $n=1$, the last summation evaluates zero),
		by plugging \eqref{eqCoeffB} and \eqref{eqCoeffB0}:
		\[
		\begin{aligned}
			(x-\lambda)g(x)&=\sum_{i'=1}^{n}b_{i'-1}x^{i'}-\sum_{i=0}^{n-1}b_i\lambda x^i\\
			&=b_{n-1}x^{n}-b_0\lambda+\sum_{i=1}^{n-1}(b_{i-1}-b_i\lambda)x^i\\
			&=a_nx^{n}+a_0+\sum_{i=1}^{n-1}a_ix^i\\
			&=f(x).
		\end{aligned}
		\]
	\end{proof}
	
	\begin{def*}[algebraically closed field]
		A field $F$ is \textbf{algebraically closed} if every non-constant polynomial in $F[x]$ has a root in $F$.
	\end{def*}
	\begin{prop}
		Given an algebraically closed field $F$, every non-constant polynomial in $F[x]$ is a product of first degree polynomials.
	\end{prop}
	\begin{proof}
		Every non-constant polynomial $f(x)\in F[x]$ has degree $n>0$. From the definition of algebraically closed field,
		$f(x)$ has a root, denoted $\lambda$.
		Therefore,
		$f(x)=(x-\lambda)g(x),$
		where $g(x)\in F[x]$.
		
		This process can be repeated until the remaining polynomial is a nonzero constant.
		Note that roots of $g(x)$ are also roots of $f(x)$.
		Since the order of $g$ is one less than that of $f$, this can be done exactly $n$ times,
		giving
		\[f(x)=k\prod_{i=1}^{n}(x-\lambda_i),\]
		where $\lambda_i\in F$ are roots of $f(x)$ and $k\in F$ is the leading coefficient.
	\end{proof}

	\subsubsection*{Eigenvalues, Eigenvectors, Schur decomposition, and Jordan form}
	
	\begin{def*}[algebraic multiplicity]
		The \textbf{algebraic multiplicity} of an eigenvalue is its multiplicity as a root of the characteristic polynomial.
	\end{def*}
	\begin{prop}
		Given a matrix $A\in\MM_n(F)$, where $F$ is an algebraically closed field,
		the sum of algebraic multiplicities is $n$.
	\end{prop}
	\begin{proof}
		Definition of algebraically closed field.
		\url{https://en.wikipedia.org/wiki/Algebraically_closed_field#Every_polynomial_is_a_product_of_first_degree_polynomials}
	\end{proof}
	
	\begin{def*}[geometric multiplicity]
		The dimension of the eigenspace associated with an eigenvalue is referred to as the eigenvalue's \textbf{geometric multiplicity}.
	\end{def*}
	\begin{thm}
		Geometric multiplicity is at least 1, at most algebraic multiplicity.
	\end{thm}
	\begin{cor}
		There is at least one eigenvalue-eigenvector pair for a square matrix over an algebraically closed field.
	\end{cor}
	\begin{proof}
		Worst case scenario: 1 eigenvalue, algebraic multiplicity is the order of the square matrix, geometric multiplicity is 1.
	\end{proof}
	\begin{cor}
		Any square matrix over an algebraically closed field is conjugate with an upper triangular matrix.
	\end{cor}
	\begin{proof}
		\url{https://math.stackexchange.com/questions/281833/matrix-similarity-upper-triangular-matrix}
	\end{proof}
	\red{The direct sum of all eigenspaces is not necessarily the original vector space!}
	
	
	\begin{prop}
		Any square matrix over an algebraically closed field is conjugate with an upper triangular matrix,
		with its eigenvalues in the main diagonal.
	\end{prop}
	\begin{proof}
		In the previous proof, a square matrix $A$ of order $n$ is transformed to
		\[
		B^{-1}AB=\begin{pmatrix}
			\lambda & C \\
			0 & A'\\
		\end{pmatrix}
		=
		\begin{pmatrix}
			\lambda & * & \dots & *\\
			0 & * & \dots & *\\
			\vdots & \vdots & \ddots & \vdots\\
			0 & * & \dots & *\\
		\end{pmatrix},
		\]
		where $A'$ is the $(n-1)\times(n-1)$ submatrix.
		The characteristic polynomial for $A$ is
		\[\det(t\idm-A)=\det(B^{-1}(t\idm-A)B)=(t-\lambda)\det(t\idm-A').\]
		Therefore the eigenvalues of $A'$ are also those of $A$, with the same algebraic multiplicities,
		except that of $\lambda$, which is decreased by 1.
	\end{proof}
	\begin{cor}
		The trace of a square matrix is the sum of its eigenvalues.
	\end{cor}
	\begin{def*}[Schur decomposition]
		Given $A\in\MM_n(K)$, where $K$ is an algebraically closed field,
		the \textbf{Schur decomposition} (or Schur’s unitary triangularization) of $A$ is
		\[A=QUQ^{-1},\]
		where $Q$ is a unitary matrix,
		and $U$ is an upper triangular matrix, called the \textbf{Schur form} of $A$.
	\end{def*}
	\begin{thm}
		Any matrix $A\in\MM_n(K)$, where $K$ is an algebraically closed field,
		has Schur decomposition.
		
		The Schur form has the same spectrum, and its eigenvalues are the diagonal entries.
	\end{thm}
	
	\begin{def*}[Jordan block]
		a \textbf{Jordan block} over a ring $R$ (whose identities are the zero 0 and one 1) is a matrix composed of zeroes everywhere except for the diagonal, which is filled with a fixed element $\lambda \in R$, and for the superdiagonal, which is composed of ones. 
	\end{def*}
	\begin{def*}[Jordan matrix]
		Any block diagonal matrix whose blocks are Jordan blocks is called a \textbf{Jordan matrix}.
	\end{def*}
	\begin{thm}
		Any $n \times n$ square matrix $A$ whose elements are in an algebraically closed field $K$ is similar to a Jordan matrix $J$,
		whose main diagonal entries are eigenvalues of $A$.
	\end{thm}
	\begin{rem*}
		Given an eigenvalue $\lambda_i$, its geometric multiplicity is the number of Jordan blocks corresponding to $\lambda_i$.
		
		The sum of the sizes of all Jordan blocks corresponding to an eigenvalue $\lambda_i$ is its algebraic multiplicity.
	\end{rem*}
	
	\subsubsection*{Spectral Theorem}
	\begin{prop}
		Any self-adjoint complex matrix is conjugate with a real diagonal matrix
		through similarity transformation by a unitary matrix.
	\end{prop}
	\begin{proof}
		Perform the Schur decomposition for the self-adjoint matrix, denoted $H$:
		\[ Q^{-1}HQ=U, \]
		where $U$ is upper triangular and $Q$ is unitary.
		Since $H$ is self-adjoint, we have
		\[ \hadj{U} = \hadj{Q}\hadj{H}\hadj{(Q^{-1})}=Q^{-1}HQ=U.\]
	\end{proof}
	
	\begin{lem}
		Unitary triangular matrix is diagonal, with diagonal entries being roots of unity.
	\end{lem}
	\begin{proof}
		If the unitary matrix is lower triangular, we take its conjugate transpose.
		The new matrix should be a unitary upper triangular matrix.
		
		Denote the unitary upper triangular matrix as $T$, then:
		
		\[(\hadj{T}T)_{ij}=\sum_{k=1}^{n}\conj{T_{ki}}T_{kj}=\delta_{ij}. \]
		
		Since $T$ is upper triangular, $T_{kj}=0,\forall k>j$. Therefore,
		\[\sum_{k=1}^{j}\conj{T_{ki}}T_{kj}=\delta_{ij}.\]
		
		Especially, for $i=1,j=1$,
		\[\conj{T_{11}}T_{11}=1,\]
		indicating $T_{11}=\ee^{i\theta_1}\ne0$ where $\theta_1\in\re$.
		
		For $i\ne1,j=1$,
		\[\conj{T_{1i}}T_{11}=0, \]
		implying $T_{1i}=0$ because $T_{11}\ne0$.
		$T_{i1}=0$ for all $i>1$ because $T$ is upper triangular.
		%		\[(\hadj{T}T)_{11}=\conj{T_{11}}T_{11}=1. \]
		
		We have proved that $T_{ij}=\ee^{i\theta_j}\delta_{ij}$ for $i<2$ or $j<2$.
		If $T_{ij}=\ee^{i\theta_j}\delta_{ij}$ for all $i< q$ or $j<q$,
		then (since $T_{kq}=0$ for all $k<q$)
		\[\delta_{pq}=(\hadj{T}T)_{pq}=\sum_{k=1}^{q}\conj{T_{kp}}T_{kq}=\conj{T_{qp}}T_{qq}. \]
		When $p=q$, this means $T_{qq}=\ee^{i\theta_q}\ne0$.
		When $p>q$, since $T_{qq}\ne0$, we have
		\[T_{qp}=\frac{\conj{\delta_{pq}}}{\conj{T_{qq}}}=0.\]
		Simultaneously, $T_{pq}$ because $p>q$ and $T$ is upper triangular.
		Therefore, $T_{ij}=\ee^{i\theta_j}\delta_{ij}$ for all $i= q$ or $j=q$.
		
		Mathematical induction concludes that $T$ is a diagonal matrix with roots of unity as diagonal entries.
		Therefore its conjugate transpose (lower triangular matrix) has the same property.
	\end{proof}
	\begin{prop}
		Any unitary matrix is conjugate with a diagonal matrix of roots of unity
		through similarity transformation by another unitary matrix.
	\end{prop}
	\begin{proof}
		Perform the Schur decomposition for the $n\times n$ unitary matrix denoted $U$:
		\[ Q^{-1}UQ=T, \]
		where \textbf{$T$ is upper triangular} and $Q$ is unitary.
		Since $U$ is unitary, we have
		\[ \hadj{T}T = \hadj{Q}\hadj{U}\hadj{(Q^{-1})}Q^{-1}UQ=Q^{-1}U^{-1}QQ^{-1}UQ=\idm.\]
		$T$ is invertible because so are $Q$ and $U$.
		Therefore \textbf{$T$ is unitary}.
		
		
		%Let $U$ be the unitary matrix, we need to prove there exists unitary matrix $V$ of the same size and diagonal matrix $D$
		%of the same size, such that
		%\[V^{-1}UV=D, D=\diag(\ee^{i\theta_1},\ee^{i\theta_2},\dots), \]
		%where $\theta_j\in\re$.
		%
		%We have proved that 
		%\[S^{-1}US=T, \]
		%where $S$ is an invertible matrix and $T$ is an upper triangular matrix whose main diagonal entries are eigenvalues of $U$.
		%Since $U$ is unitary:
		%\[ \hadj{T}T= \]
	\end{proof}
	
	\subsubsection*{Main Text}

	\paragraph{P45}
	\subparagraph{Spectral theorem}
	\blue{Note that a general (invertible) matrix cannot be diagonalized}, but it can be similarity-transformed to upper triangular form,
	where the main diagonal entries are eigenvalues.
	
	Since similarity transformation does not change the trace of a matrix, the trace is still the sum of eigenvalues.
	
	\section{Lie Algebras and Lie Algebra Representations}
	\subsection*{5.2 Lie algebras of the orthogonal and unitary groups}
	\paragraph{P52}
	
	$U(n)$, different from $O(n)$, is connected because
	\[\ee^{i\pi}=-1, \]
	while this is impossible in $\re$.
	
	There are two components in $O(n)$, one of which containing identity is $SO(n)$.
	
	\paragraph{P53}
	
	\red{Prove: Skew-Hermitian matrices in $\mathfrak{u}(n)$ are diagonalizable.}
	
	$U(n)$ and $\mathfrak{u}(n)$ matrix are diagonalizable by unitary matrix, %$\therefore U(n)$ is connected,
	so any element $\Omega$ in the Lie group $U(n)$
	can be written as
	\[\Omega=\ee^{tX}, \]
	where $t\in\re$ and $X$ is in the corresponding Lie algebra.
	
	Similarly $SO(n)$ can be written as exponential, but $SL(n)$ is large to be expressed this way.
	(maybe only Jordan form?)
	
	\subsection*{5.4 Lie algebra representations}
	\paragraph{P54}
	Lie algebra representation is algebra homomorphism because it is linear map + preserving Lie bracket.
	
	\paragraph{P57}
	
	The adjoint Lie algebra representation is a regular representation.
	
	\subsection*{5.5 Complexification}
	\paragraph{P61}
	The complexification of $\mathfrak{su}(2)$ (3-dimension) is $\mathfrak{sl}(2,\co)$ (6-real dimension).
	
	The complexification of $\mathfrak{gl}(n,\co)$ is a copy of two.
	
	\section{The Rotation and Spin Groups in 3 and 4 Dimensions}
	\subsection*{6.1 The rotation group in three dimensions}
	\paragraph{P64}
	
	\subparagraph{Specialty in 3 dimension}
	Something very special that happens for orthogonal groups only in dimension
	$n = 3$ is that the vector representation (the defining representation of $\operatorname{SO}(n)$
	matrices on $\re^n$) is isomorphic to the adjoint representation.
	
	This is because any plane is bijectively corresponding to its perpendicular vector in 3 dimension.
	
	\subparagraph{Difference between Lie group representation and Lie algebra representation}
	Note that the adjoint Lie group representation is
	\[(Ad,\mathfrak{g})\]
	while the adjoint Lie algebra representation is
	\[ (ad, \mathfrak{g}). \]
	
	\subparagraph{Pi vector}
	Since the vector representation $(\pi_{vector}, \re^3
	)$ on column
	vectors and the adjoint representation $(Ad, \mathfrak{so}(3))$
	are isomorphic, you can use the same matrix (in the adjoint representation)
	to represent $g$ (which usually is represented as a column vector).
	
	But the vector and matrix representations are transformed (or acting on other objects) differently (see top of P65).
\end{document}
